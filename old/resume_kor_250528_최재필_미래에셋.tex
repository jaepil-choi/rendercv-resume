\documentclass[10pt, letterpaper]{article}

% Packages:
\usepackage{kotex}
\usepackage[
    ignoreheadfoot, 
    top=1 cm, 
    bottom=1 cm, 
    left=1 cm, 
    right=1 cm, 
    footskip=0.5 cm, 
    % showframe
]{geometry} 
\usepackage{titlesec} 
\usepackage{tabularx} 
\usepackage{array} 
\usepackage[dvipsnames]{xcolor} 
\definecolor{primaryColor}{RGB}{0, 0, 0} 
\usepackage{enumitem} 
\usepackage{fontawesome5} 
\usepackage{amsmath} 
\usepackage[
    pdftitle={Jaepil Choi's CV},
    pdfauthor={Jaepil Choi},
    pdfcreator={LaTeX with RenderCV},
    colorlinks=true,
    urlcolor=primaryColor
]{hyperref} 
\usepackage[pscoord]{eso-pic} 
\usepackage{calc} 
\usepackage{bookmark} 
\usepackage{lastpage} 
\usepackage{changepage} 
\usepackage{paracol} 
\usepackage{ifthen} 
\usepackage{needspace} 
\usepackage{iftex} 

% Ensure that generate pdf is machine readable/ATS parsable:
\ifPDFTeX
    \input{glyphtounicode}
    \pdfgentounicode=1
    \usepackage[T1]{fontenc}
    \usepackage[utf8]{inputenc}
    \usepackage{lmodern}
\fi

\usepackage{charter}

% Some settings:
\raggedright
\AtBeginEnvironment{adjustwidth}{\partopsep0pt} 
\pagestyle{empty} 
\setcounter{secnumdepth}{0} 
\setlength{\parindent}{0pt} 
\setlength{\topskip}{0pt} 
\setlength{\columnsep}{0.15cm} 
\pagenumbering{gobble} 

\titleformat{\section}{\needspace{4\baselineskip}\bfseries\large}{}{0pt}{}[\vspace{1pt}\titlerule]

\titlespacing{\section}{
    -1pt
}{
    0.3 cm
}{
    0.2 cm
}

\renewcommand\labelitemi{$\vcenter{\hbox{\small$\bullet$}}$}

\newenvironment{highlights}{
    \begin{itemize}[
        topsep=0.10 cm,
        parsep=0.10 cm,
        partopsep=0pt,
        itemsep=0pt,
        leftmargin=0 cm + 10pt
    ]
}{
    \end{itemize}
}

\newenvironment{highlightsforbulletentries}{
    \begin{itemize}[
        topsep=0.10 cm,
        parsep=0.10 cm,
        partopsep=0pt,
        itemsep=0pt,
        leftmargin=10pt
    ]
}{
    \end{itemize}
}

\newenvironment{onecolentry}{
    \begin{adjustwidth}{
        0 cm + 0.00001 cm
    }{
        0 cm + 0.00001 cm
    }
}{
    \end{adjustwidth}
}

\newenvironment{twocolentry}[2][]{
    \onecolentry
    \def\secondColumn{#2}
    \setcolumnwidth{\fill, 4.5 cm}
    \begin{paracol}{2}
}{
    \switchcolumn \raggedleft \secondColumn
    \end{paracol}
    \endonecolentry
}

\newenvironment{threecolentry}[3][]{
    \onecolentry
    \def\thirdColumn{#3}
    \setcolumnwidth{, \fill, 4.5 cm}
    \begin{paracol}{3}
    {\raggedright #2} \switchcolumn
}{
    \switchcolumn \raggedleft \thirdColumn
    \end{paracol}
    \endonecolentry
}

\newenvironment{header}{
    \setlength{\topsep}{0pt}\par\kern\topsep\centering\linespread{1.5}
}{
    \par\kern\topsep
}

\newcommand{\placelastupdatedtext}{%
  \AddToShipoutPictureFG*{
    \put(
        \LenToUnit{\paperwidth-1 cm-0 cm+0.05cm},
        \LenToUnit{\paperheight-0.5 cm}
    ){\vtop{{\null}\makebox[0pt][c]{
        \small\color{gray}\textit{Last updated in November 2024}\hspace{\widthof{Last updated in November 2024}}
    }}}%
  }%
}

\let\hrefWithoutArrow\href

\begin{document}
    \newcommand{\AND}{\unskip
        \cleaders\copy\ANDbox\hskip\wd\ANDbox
        \ignorespaces
    }
    \newsavebox\ANDbox
    \sbox\ANDbox{$|$}

    \begin{header}
        \fontsize{25 pt}{25 pt}\selectfont 최재필

        \vspace{5 pt}

        \normalsize
        \mbox{서울, 대한민국}%
        \kern 5.0 pt%
        \AND%
        \kern 5.0 pt%
        \mbox{\hrefWithoutArrow{mailto:chljeffreyz@gmail.com}{chljeffreyz@gmail.com}}%
        \kern 5.0 pt%
        \AND%
        \kern 5.0 pt%
        \mbox{\hrefWithoutArrow{tel:+82-10-2589-5000}{010-2589-5000}}%
    \end{header}

    \vspace{5 pt - 0.3 cm}


    \section{Education \& Certificates}

        
        \begin{twocolentry}{
            2024년 3월 – 현재
        }
            \textbf{KAIST 경영대학}, 금융공학 석사\end{twocolentry}



        \vspace{0.2 cm}

        \begin{twocolentry}{
            2023년 1월 – 현재
        }
            \textbf{WorldQuant Online University}, 금융공학 석사\end{twocolentry}



        \vspace{0.2 cm}

        \begin{twocolentry}{
            2013년 3월 – 2020년 8월
        }
            \textbf{성균관대학교}, (주) 글로벌경제 / (부) 인포매틱스 학사\end{twocolentry}



        \vspace{0.2 cm}

        \begin{onecolentry}
            \textbf{금융 자격증}\end{onecolentry}

        \vspace{0.10 cm}
        \begin{onecolentry}
            \begin{highlights}
                \item CFA: 1차 합격 (2019)
                \item KOFIA: 투자자산운용사 (2019), 파생상품투자권유자문인력 (2023), 펀드투자권유자문인력 (2022)
            \end{highlights}
        \end{onecolentry}



    
    \section{Work Experience}

        \begin{twocolentry}{
            2025년 1월 – 현재
        }
            \textbf{메리츠증권}, 트레이딩본부 매크로트레이딩팀 (파트타임, 학업 병행) -- 서울, 대한민국\end{twocolentry}

        \vspace{0.10 cm}
        \begin{onecolentry}
            \begin{highlights}
                \item 내부 딜러들이 코딩 없이 글로벌 매크로(채권/주식인덱스/원자재/FX) 전략을 엑셀 함수 입력하듯 만들 수 있는 플랫폼 개발 중
                \item 다양한 객체지향 디자인패턴을 도입하여 전략/유틸 기능 추가가 용이하고 데이터 쿼리, 전/후처리 등이 설정파일(config) 조작만으로 적용/수정될 수 있도록 확장성을 고민해 설계
            \end{highlights}
        \end{onecolentry}


        \vspace{0.2 cm}
        
        \begin{twocolentry}{
            2024년 7월 – 2024년 8월
        }
            \textbf{Zero One AI}, 금융 리서치 하계 인턴 -- 서울, 대한민국\end{twocolentry}

        \vspace{0.10 cm}
        \begin{onecolentry}
            \begin{highlights}
                \item AQR 'Replication Crisis' 논문 기반 CRSP \& Compustat (WRDS)를 활용한 퀀트 팩터 DB 구축 및 대량 API 문서 전처리를 통한 일별 팩터 데이터 생산용 확장 가능한 API 래퍼 개발
            \end{highlights}
        \end{onecolentry}


        \vspace{0.2 cm}

        \begin{twocolentry}{
            2022년 7월 – 2024년 2월
        }
            \textbf{우리은행}, 마이데이터 사업부 서비스 기획자 -- 서울, 대한민국\end{twocolentry}

        \vspace{0.10 cm}
        \begin{onecolentry}
            \begin{highlights}
                \item 마이데이터 서비스 내 '내 투자 스토리' 개인화 데이터 큐레이션 기능 개발 주도. 스토리보드 작성, SQL/Python 기반 EDA 수행 및 기능 출시 후 푸시 알림 응답률 11\% 달성(1만 건 이상 발송 푸시 중 최고)
                \item 출시 후 A/B 테스트를 통해 KPI 개선 효과 통계적 검증
                \item 가입 관련 KPI 정의, SQL 활용 데이터 추출, 관리자 대시보드 설계를 통한 성과 모니터링 체계 구축
            \end{highlights}
        \end{onecolentry}


        \vspace{0.2 cm}

        \begin{twocolentry}{
            2021년 9월 – 2022년 3월
        }
            \textbf{우리은행}, 지점 근무 -- 서울, 대한민국\end{twocolentry}

        \vspace{0.10 cm}
        \begin{onecolentry}
            \begin{highlights}
                \item Excel 기반 기업대출 KPI 계산기를 개발하여, 지점장님의 기업 대출 실행 여부 의사결정 지원
                \item 사내 제안광장에 30건 이상의 시스템·프로세스 개선안을 올려 우수 제안상 2위 수상 (입사 후 3개월 이내)
            \end{highlights}
        \end{onecolentry}


        \vspace{0.2 cm}

        \begin{twocolentry}{
            2020년 6월 – 2020년 11월
        }
            \textbf{HaaFor Research Korea}, 퀀트 리서치 인턴 -- 서울, 대한민국\end{twocolentry}

        \vspace{0.10 cm}
        \begin{onecolentry}
            \begin{highlights}
                \item 미국 주식 4,000+ 종목에 대한 대체 데이터 리서치 및 특허·뉴스·온라인 지표 활용 신규 데이터셋 구축해 다른 퀀트 리서쳐들의 상관관계 낮은 알파 전략 개발 지원
                \item 일단위 리밸런싱 주기의 시장/섹터/팩터 중립적인 롱숏 알파 전략 개발 경험
            \end{highlights}
        \end{onecolentry}



    
    \section{Projects}

        
        \begin{onecolentry}
            \textbf{kor-quant-dataloader}\end{onecolentry}

        \vspace{0.10 cm}
        \begin{onecolentry}
            \begin{highlights}
                \item 생존 편향을 제거하고 데이터 로드 편의성을 높인 한국 주식 데이터셋 파이썬 패키지 개발 (KRX 데이터)
                \item 날짜 범위만 지정하면 long format: 날짜-종목 인덱스, 각종 팩터 컬럼 형식 또는 wide format: 날짜 인덱스, 종목 컬럼 형식으로 데이터 반환
            \end{highlights}
        \end{onecolentry}


        \vspace{0.2 cm}

        \begin{onecolentry}
            \textbf{qtrsch}\end{onecolentry}

        \vspace{0.10 cm}
        \begin{onecolentry}
            \begin{highlights}
                \item 한국 주식 데이터로 파마-프렌치 5요인 모형 직접 구축 후 FnGuide 벤치마크와 팩터 수익률 검증
                \item 팩터 기반 요소로 수익률 분해, Group Neutralization 효과 분석
                \item 한국 시장에서 PEAD(실적발표 후 주가이상현상) 현상 심층 분석 및 실적 발표 전후 누적 수익률 시각화
            \end{highlights}
        \end{onecolentry}


        \vspace{0.2 cm}

        \begin{onecolentry}
            \textbf{학술 논문 구현: 텍스트 마이닝을 활용한 금융통화위원회 의사록 분석 (2019)}\end{onecolentry}

        \vspace{0.10 cm}
        \begin{onecolentry}
            \begin{highlights}
                \item 텍스트 마이닝 기법으로 금융통화위원회 의사록의 톤(매파/비둘기파)을 측정하고 정책금리 변동과의 상관관계 분석
            \end{highlights}
        \end{onecolentry}



    
    \section{Additional Information}

        
        \begin{onecolentry}
            \textbf{기술:} Python, Oracle SQL, Git-flow, 웹 스크래핑, MS-Office
        \end{onecolentry}

        \vspace{0.2 cm}

        \begin{onecolentry}
            \textbf{언어:} 영어 (TOEIC 985, TOEIC Speaking 180), 독일어 (Zertifikat Deutsch B1)
        \end{onecolentry}

        \vspace{0.2 cm}

        \begin{onecolentry}
            \textbf{병역:} 의무경찰 병장 만기전역 (2014년 9월 - 2016년 6월)
        \end{onecolentry}


    

\end{document}